%-----------------------------------------------------------------------------
% Author: Rayla Kurosaki
% GitHub: https://github.com/rkp1503
%-----------------------------------------------------------------------------
\documentclass{rayla-project}
\usepackage{rayla-style}

%-----------------------------------------------------------------------------
% Project Information
%-----------------------------------------------------------------------------
\myUniversity{
    Rochester Institute of Technology\\
    College of Science\\
    School of Mathematical Sciences
    }
\myTitle{Red Blood Cell Production}
\myName{Ramsey (Rayla) Phuc}
\courseid{MATH-421.01}
\courseName{Mathematical Modeling}
\professorName{Dr. Nathan Cahill}
\term{2020-2021 Fall}
\dueDate{December 7th, 2020}

%-----------------------------------------------------------------------------
% Start of Document
%-----------------------------------------------------------------------------
\begin{document}

    \maketitle

    %-------------------------------------------------------------------------
    % Abstract
    %-------------------------------------------------------------------------
    \begin{abstract}
        Red Blood Cells are present in the human body and their purpose is to deliver oxygen to the human body while giving carbon dioxide for humans to exhale. The purpose of this paper is to show that Red Blood Cells can maintain an equilibrium state if a certain condition is met. Multiple models have been tested to show that this condition is consistent given the proper assumptions. The first model being analyzed is a system of linear difference equations. The second model being analyzed is a system of linear differential equations. The third model being analyzed is also a system of linear difference equations with the assumption that the spleen removes a portion of Red Blood Cells based on the number of Red Blood Cells present in the human body. In particular, as the number of Red Blood Cells in the human body increases, the portion of Red Blood Cells filtered out by the spleen will increase non-linearly. This condition is where $\gamma = 1$.
    \end{abstract}

    %-------------------------------------------------------------------------
    % Body
    %-------------------------------------------------------------------------
    \input{contents/chapter01.tex}
    \section{Problem Statement}

The circulatory system in the human body produces and destroys Red Blood Cells every day. The problem is to determine the number of Red Blood Cells on a particular day. \cite{book1}

    \section{Initial Assumptions and Simplifications}

\begin{enumerate}
\item Spleen filters out and destroys a certain fraction of the Red Blood Cells every day.
\item Bone marrow produces a number of Red Blood Cells that is proportional to the number of Red Blood Cells lost on the day prior.
\end{enumerate}

    \section{Initial Conditions and Definitions}

\begin{itemize}
\item $n$ represents a particular day.
\item $R_n$ represents the number of Red Blood Cells circulating in the blood on day $n$.
\item $M_n$ represents the number of Red Blood Cells produced by the bone marrow on day $n$.
\item $f$ represents the fraction of Red Blood Cells the spleen removes.
\item $\gamma$ represents the number of Red Blood Cells produced per number of Red Blood Cells lost.
\end{itemize}

    \input{contents/chapter05.tex}
    \section{Conclusion}

What we have done is shown that the only possible way for Red Blood Cell population to maintain in an equilibrium state is when $\gamma = 1$. An idea that can be extended from this work is to see what would happen if blood loss occurred at some day $n$ and how the state of the Red Blood Cell population would be affected if a small or a large portion of that population has been removed from the body.


    %-------------------------------------------------------------------------
    % Bibliography
    %-------------------------------------------------------------------------
    \newpage
    % \nocite{*}
    \printbibliography[title={Bibliography}]

\end{document}
